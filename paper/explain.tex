\documentclass{article}
\usepackage{amsmath,graphicx, bbm}
\usepackage{hyperref}

\title{Metric explain}

\author{Changwoon Choi}

\begin{document}
\maketitle
\section{metrics}
\begin{itemize}
\item \textbf{1-NNA}, proposed by ~\cite{1-nna}, evaluates whether two distributions of distributions are identical.
	For distribution $X$, we denote the nearest neighbor as $N_X = \text{argmin}_{Y \in S_{-X}}{d(X, Y)} $, where $S_{-X}$ represents the set including the entire generated and reference distributions except itself, $S_{-X} = S_r \cup S_g - {X}$.
	1-NNA is defined to be
	\begin{equation}
		\text{1-NNA}(S_g, S_r) = \frac{
			\sum_{X \in S_g}{\mathbbm{1}_{N_X \in S_g}} + \sum_{Y \in S_r}{\mathbbm{1}_{N_Y \in S_r}}
		}{|S_g| + |S_r|},
	\end{equation}
	where $\mathbbm{1}$ is an indicator function.
	The optimal value of 1-NNA is 50\%, when the distribution of two sets are equal, unable to distinguish the two sets.
\item \textbf{COV} measures the proportion of shapes in the reference set that are matched to at least one distribution in the generated set, formally defined by
\begin{equation}
	\text{COV}(S_g, S_r) = \frac{
		|\{\text{argmin}_{Y \in S_r} d(X, Y) | X \in S_g\}|
	}{|S_r|}.
\end{equation}
\item \textbf{MMD} measures the quality of the generated set.
	For each distribution in the reference set, we compute the distance to nearest neighbor and average it, which is formally defined by
	\begin{equation}
		\text{MMD}(S_g, S_r) = \frac{1}{|S_r|} \sum_{Y \in S_r}{\min_{X \in S_g}{d(X, Y)}}
	\end{equation}

\end{itemize}

\bibliographystyle{IEEEbib}
\bibliography{our_bib}

\end{document}