\documentclass{article}
\usepackage{spconf,amsmath,graphicx, bbm}
\usepackage{tabularx}
\usepackage{multicol} 
\usepackage{multirow}
\usepackage{booktabs}
\usepackage{hyperref}
\usepackage{url}
\usepackage[toc,page]{appendix}
\usepackage{subfig}
\usepackage{subcaption}
\usepackage{ragged2e}
\captionsetup{compatibility=false}
\title{the methodology for stereo image learning representation}

\name{Seongrae Kim$^{\star}$ \qquad Changwoon Choi$^{\dagger}$ \qquad Sangwoo Han$^{\ddagger}$}
\address{$^{\star}$ Mechanical Engineering, Seoul National University \\ $^{\dagger}$ Electrical \& Computer Engineering, Seoul National University \\$^{\ddagger}$ Earth \& Environmental Sciences, Seoul National University}

\begin{document}
\maketitle

\begin{abstract}
Automating audio generation and synthesis is a key building block of advanced computer listening applications such as auto composition. Whilst various audio generative neural networks exist that target producing audio with both high-fidelity and global structure, far less attention has been paid to generate stereo audio, which is essential to satisfy the listeners. In this paper, we propose a simple but effective training scheme for stereo audio generation task by reparameterizing left-right channels to Mid-Side channels. To address this problem, we first introduce a novel public dataset\footnote{Dataset and code are available on \url{https://github.com/changwoonchoi/ml2020}} which features 960 short high-fidelity stereo audios. We also propose new representations, namely \textit{`Side Distance'} and \textit{`Short-time Side Distance'} that effectively capture the stereo image of stereo audio. Our results clearly show that proposed method is superior to the conventional method in generating stereo audio samples on both quantitative and qualitative evaluations.
\end{abstract}

\begin{keywords}
Generative model, Multi-channel audio, Learning representation
\end{keywords}

\section{Introduction}
\label{sec:intro}
Blahblah

Overall, we propose an $\cdot$. We show promising results on stereo audio generation problem with our new dataset. In summary, the main contributions of this work are:
\begin{itemize}
    \item 
    \item 
    \item 
\end{itemize}
\section{Related Work}
\label{sec:related_work}
\subsection{Audio Generative Model}

The earliest audio-generated models for audio tends to focus on speech synthesis. These datasets require handling variable length, and WaveNet~\cite{wavenet} used autoregressive method for variable length inputs and outputs. A flow-based WaveGlow~\cite{waveglow} has emerged that compensates for slow speed of autoregressive models. In comparison to speech, audio generation for music is relatively in the development stage. ~\cite{nsynth} proposed to use WaveNet for generate single musical note, but it was still slow and global latent conditioning was impossible. GANSynth~\cite{gansynth} with Generative Adversarial Network solves these shortcomings and provides high-fidelity and locally-coherent audio by modeling log magnitudes and instantaneous frequencies with sufficient frequency resolution in the spectral domain. With the current technology, it is possible to learn a single note of a variety of instruments to convincingly describe the sound of a real instrument, and to create a variety of synthesizers for interpolation between two instruments. In spite of such breakthrough development, the reason that the generated musical notes cannot be used for commercial music is that the generated result is a single channel. Despite this breakthrough, the generated musical notes cannot replace commonly used virtual instruments due to the limitation of a single channel.


\subsection{Multi-Channel Audio}

In deep learning, multi-channel settings are mainly used for speech separation~\cite{multi_ch_sourcesep}, speech enhancement~\cite{speech_enh}, and speech recognition~\cite{speech_recog} due to their ability to utilize information about speech source location. On the other hand, 
\if 0
Fink et al.(2015) 
\fi
~\cite{upmix} proposed a upmixing conversion of the mono signal to pseudo stereo in order to enhance the audio effect. However, as far as we know, there are still no attempts to generate high-fidelity multi-channel audio. It is thought that it is necessary to learn the difference while maintaining the coherency of both channels to form a spatial sense of sound. A technique referred to as “mid-side coding” exploits the common part of a stereophonic input signal by encoding the sum and difference signals of the two input signals rather than the input signals themselves.(~\cite{mscoding}) Therefore, for a high-performance multi-channel audio generative model in the future, we would like to train the GANSynth baseline through the mid-side coding to verify whether this attempt is able to effectively learn stereo image.
\section{Proposed Method}
\label{sec:method}

\subsection{Training scheme}
\label{subsec:trainingscheme}
\begin{figure}[t]
    \centering
    \includegraphics[width=0.85\linewidth]{assets/figures/LRMS.pdf}
    \caption{Example of channel representation for two-channel audio. The left one is the L-R channel representation, and the right one is the M-S channel representation.}
    \label{fig:quality_result}
\end{figure}
As we mentioned above, channel coherency is an important property for the plausible spatiality formation in two-channel audio. However, when the network generates two-channel audio, if left and right channels are created without a guide of channel coherency, the spatiality of the generated audio will not be appropriate. Since creating additional networks associated with channel coherence caused overhead, it is recommended to find a different method to avoid burdening the network. Therefore, it is valid to make the following conjecture: Network will learn stereo image better when using the mid-side (M-S) channel than using the left-right (L-R) channel. When $y$ is stereo audio and $y_M$ and $y_S$ are mid and side channel of stereo audio $y$, $y_M$ and $y_S$ as following:
\begin{equation}
    y_M = {\frac{y_L + y_R}{2}},\quad
    y_S = {\frac{y_L - y_R}{2}}\nonumber
\end{equation}
where $y_L$ and $y_R$ are left and right channel of stereo audio $y$.

\subsection{Custom dataset}
\label{subsec:dataset}
To validate the aforementioned methodology for the stereo image learning for the network, we needed a new dataset. Since we focus on a stereo audio generation model, we needed an audio set with the drastic and various stereo images for our conjecture, but none of the conventional datasets were appropriate. Inspired by the NSynth dataset, which is mainly used by previous studies on neural audio synthesis ~\cite{gansynth, nsynth}, we composed the new dataset using the stab, which is a single staccato note or chord that adds dramatic punctuation to a composition, used in modern electronic music. For clarity of the task, we have adjusted the length of each sample to 400ms. We configured a dataset with a sample rate of 44.1kHz and a bit depth of 16bit.

Since the lack of a large number of our sources, we performed three augmentations to obtain a sufficient amount of data. The augmentation performed are as follows: L-R channel change, time-stretching without pitch shift, and FIR filtering. We produced a total of 11,520 data that maintain the characteristics of the original data through these augmentations. Note that we did not add variations on the pitch because the data often have atonal properties.
\section{Experimental Results}
\label{sec:experiment}

\section{Conclusion}
\label{sec:conclusion}
In this paper, we propose a novel, yet simple, training scheme for a stereo audio generative model. Also, we introduce a new dataset composed of two-channel stereo audios with rich stereo images. By reparameterizing the L-R channel into the M-S channel, the experiments on the proposed dataset demonstrate that our proposed training scheme gives promising results on every evaluation metrics (MMD, COV, 1-NNA). Furthermore, the study on human evaluations shows that our method is superior to the conventional method in terms of audio quality.
\bibliographystyle{IEEEbib}
\bibliography{our_bib}
% \newpage
\clearpage
\appendix
% \section*{Appendix}
\section{Additional Implementation Details}
In this section we provide several additional details that are not provided in the main sections.
\subsection{Neural Network Architecture}
\subsection{}
\section{Dataset Details}
\section{Evaluation Metrics}
\begin{table}[t]
\end{table}
\end{document}
