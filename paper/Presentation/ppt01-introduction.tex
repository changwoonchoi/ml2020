\section{Introduction}
\subsection{Problem Formulation}
\begin{frame}{Problem Formulation}
    \begin{itemize}
        \item Motivation
        \begin{itemize}
            \item To satisfy the modern audience, at least a stereo channel environment must be procided.\\
            However most of neural audio synthesis models did not focus on multichannel audio generation.
        \end{itemize}
        \bigskip
        \item Objective
        \begin{itemize}
            \item We want to make a \textbf{generative model} that produces short \textbf{stereo} audio (2-channel audio) that has good \textbf{`stereo image'}.
        \end{itemize}
        \bigskip
        \item Formally,
        \begin{itemize}
            \item Input: random noise, condition vector
            \item Output: short stereo audio (<1s)
        \end{itemize}
    \end{itemize}
\end{frame}

\subsection{Conjecture}
\begin{frame}{Conjecture}
    \begin{center}
        \textit{``It will learn stereo image better when using \textbf{M-S} channel than using L-R channel.''}
    \end{center}
    \bigskip
    \bigskip
    \smallskip
    \begin{enumerate}
        \item The stereo images of the L-R channel has coherency in both channels.
        \item When learning stereo images using M-S channel for two channels, there is no need to train coherency on both sides.
    \end{enumerate}
    \bigskip
    $\therefore$ We would like to verify the conjecture above by comparing the results of train by L-R channel and M-S channel.
\end{frame}

\subsection{Contribution}
\begin{frame}{Contribution}
    \begin{itemize}
        \item 
        \bigskip
        \item
        \bigskip
        \item
    \end{itemize}
\end{frame}