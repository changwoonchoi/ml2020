\section{Introduction}
\subsection{Problem Formulation}
\begin{frame}{Problem Formulation}
    \begin{itemize}
        \item Motivation
        \begin{itemize}
            \item To satisfy the modern audience, at least a stereo channel environment must be provided.\\
            However most of neural audio synthesis models did not focus on multichannel audio generation.
        \end{itemize}
        \bigskip
        \item Objective
        \begin{itemize}
            \item We want to verify a methodology that enables the network to learn \textbf{`stereo images'} of the target \textbf{stereo audio}.
        \end{itemize}
        \bigskip
    \end{itemize}
\end{frame}
\if 0
        \item Formally,
        \begin{itemize}
            \item Input: random noise
            \item Output: short stereo audio (<1s)
        \end{itemize}
\fi
\subsection{Conjecture}
\begin{frame}{Conjecture}
    \begin{center}
        \textit{``It will learn stereo image better when using \textbf{Mid-Side} (M-S) channel than using Left-Right (L-R) channel.''}
    \end{center}
    \bigskip
    \bigskip
    \smallskip
    \begin{enumerate}
        \item The stereo images of the L-R channel has coherency in both channels.
        \item When learning stereo images using M-S channel for two channels, there is no need to train coherency on both sides.
    \end{enumerate}
    \bigskip
    $\therefore$ We would like to verify the conjecture above by comparing the results of train by L-R channel and M-S channel.
\end{frame}
\if 0
\subsection{Contribution}
\begin{frame}{Contribution}
    \begin{itemize}
        \item The first methodology for ensuring that the generation model focuses on sampling audios that have high-fidelity and diverse stereo images.
        \bigskip
        \item We propose a simple novel method to train 2 channel audio by reparameterizing $L-R$ to $M-S$
        \bigskip
        \item The introduction of a new public dataset for stereo audio that has high-fidelity and various stereo images.
    \end{itemize}
\end{frame}
\fi
\begin{frame}{Contribution}
    \begin{itemize}
        \item We propose a simple, yet effective, training scheme for stereo audio generation task.
        \bigskip
        \item The introduction of a new public dataset composed of two channel stereo audios which have high-fidelity and rich stereo images.
        \bigskip
        \item We introduce a proper representation of stereo image, namely `Side distance' and `STSD', and define the distance metric between them, enabling quantitative evaluation of the stereo image generation model.
    \end{itemize}
\end{frame}